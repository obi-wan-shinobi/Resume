%%%%%%%%%%%%%%%%%%%%%%%%%%%%%%%%%%%%%%%%%
% Medium Length Professional CV
% LaTeX Template
% Version 2.0 (8/5/13)
%
% This template has been downloaded from:
% http://www.LaTeXTemplates.com
%
% Original author:
% Trey Hunner (http://www.treyhunner.com/)
%
% Important note:
% This template requires the resume.cls file to be in the same directory as the
% .tex file. The resume.cls file provides the resume style used for structuring the
% document.
%
%%%%%%%%%%%%%%%%%%%%%%%%%%%%%%%%%%%%%%%%%%%%%%%%%%%%%%%%%%%%%%%%%%%%%%%%%%%%%%%%%%
% Medium Length Professional CV
% LaTeX Template
% Version 2.0 (8/5/13)
%
% This template has been downloaded from:
% http://www.LaTeXTemplates.com
%
% Original author:
% Trey Hunner (http://www.treyhunner.com/)
%
% Important note:
% This template requires the resume.cls file to be in the same directory as the
% .tex file. The resume.cls file provides the resume style used for structuring the
% document.
%
%%%%%%%%%%%%%%%%%%%%%%%%%%%%%%%%%%%%%%%%%

%----------------------------------------------------------------------------------------
%	PACKAGES AND OTHER DOCUMENT CONFIGURATIONS
%----------------------------------------------------------------------------------------

\documentclass{resume} % Use the custom resume.cls style

\usepackage{hyperref}
\usepackage[left=0.45in,top=0.3in,right=0.45in,bottom=0.3in]{geometry} % Document margins

  % If using pdflatex:
\usepackage[utf8]{inputenc}
\usepackage[T1]{fontenc}
\usepackage[default]{lato}
\usepackage{fontawesome}
\usepackage{setspace}
\usepackage{biblatex}
\usepackage{dashrule}
\usepackage{graphicx}
\usepackage{tikz}
\usepackage{hyperref}
\hypersetup{
    colorlinks=true,
    linkcolor=blue,
    filecolor=magenta,
    urlcolor=cyan,
}

\newcommand*\googlescholaricon{\raisebox{-0.3em}{\includegraphics[width=1em]{google-scholar.pdf}}}

\newcommand{\tab}[1]{\hspace{.2667\textwidth}\rlap{#1}}
\newcommand{\itab}[1]{\hspace{0em}\rlap{#1}}
\name{Shreyas Kalvankar} % Your name
% \address{\faMapMarker ~ Nashik, Maharashta, India} % Your address
%\address{123 Pleasant Lane \\ City, State 12345} % Your secondary addess (optional)
\address{\footnotesize \faPhone ~ (+91)~9423555723 ~ \faEnvelope ~ shreyaskalvankar@gmail.com ~ 
\faGlobe ~ obi-wan-shinobi.github.io}

\address{\small \faLinkedin ~ \href{https://linkedin.com/in/shreyas-kalvankar}{linkedin.com/in/shreyas-kalvankar} ~ \faGithub ~ \href{https://github.com/obi-wan-shinobi}{github.com/obi-wan-shinobi}}

% \googlescholaricon ~ \href{https://scholar.google.com/citations?hl=en&user=W8-T__UAAAAJ}{Scholar}

\addbibresource{references.bib}

\begin{document}

%----------------------------------------------------------------------------------------
%	EDUCATION SECTION
%----------------------------------------------------------------------------------------
\smallskip
\begin{rSection}{Education}

{\bf Bachelor of Engineering (Computer Engineering)} \hfill {\em 2017 - 2021}
\\ K.K. Wagh Institute of Engineering  \hfill { Overall GPA: 9.7/10}
\\ Education \& Research, Nashik    \hfill {\small{(Rank 1)}}

\smallskip

% {\bf Higher Secondary Certificate} \hfill {\em 2017}
% \\ HPT Arts \& RYK Science College, Nashik \hfill { Percentage: 87.07\%}

\end{rSection}
\begin{rSection}{PUBLICATIONS}
\vspace{-1em}
\item Shreyas Bapat et al. \textit{EinsteinPy: A Community Python Package for General Relativity.} 2020.\\ arXiv: \href{https://arxiv.org/abs/2005.11288}{2005.11288 [gr-qc]}.

\item Shreyas Kalvankar et al. \textit{Galaxy Morphology Classification using EfficientNet Architectures.} 2020.\\ arXiv:
\href{https://arxiv.org/abs/2008.13611}{2005.13611 [cs.CV]}.

\item Kalvankar, Sh., Pandit, Hr., Parwate, Pr., Patil, At. \& Kamalapur, Sn., (2022). \textit{Astronomical Image Colorization and Up-scaling with Conditional Generative Adversarial Networks.}\\ In: Demmler, D., Krupka, D. \& Federrath, H. (Hrsg.), INFORMATIK 2022. Gesellschaft für Informatik, Bonn. (S. 489\-498).
\href{https://dl.gi.de/bitstream/handle/20.500.12116/39540/mlastro\_04.pdf}{DOI: 10.18420/inf2022\_40}.
\end{rSection}

\smallskip
%----------------------------------------------------------------------------------------
%	WORK EXPERIENCE SECTION
%----------------------------------------------------------------------------------------
\smallskip
\begin{rSection}{Professional Experience}

\begin{rSubsection}{Dalton Maag Ltd.}{November 2021 - Present}{}{}
\begin{rSubsubsection}{\textbf{Software Developer}}{London, United Kingdom}{}
\item \textbf{CJK Project}: Chinese, Japanese, and Korean (CJK) typefaces require designing thousands of glyphs manually, which is extremely time-consuming. My manager and I designed a system for a POC in Python for the automatic generation of CJK font glyphs using Genetic Algorithms, which would significantly reduce the production time.
\item \textbf{Pricebot}:
\item Pricebot simplified typeface pricing by considering factors like the number of weights, axes, and scripts. It aimed to ensure accurate and consistent pricing while relieving designers of the time-consuming and error-prone task of manual quoting, thus preventing project overruns and unexpected costs.
\item It's a web app with a Ruby on Rails back-end and a VueJS \& Typescript front-end. I developed and fine-tuned the pricing models in Typescript based on expected outputs.
\item My colleague and I enhanced glyph data models for Arabic, Greek, Cyrillic scripts, accurately representing their letters. I also created a new model from scratch for Devanagari, ensuring precise pricing for non-Latin projects.
\item Using graph theory, I devised a process to efficiently create project plans that accurately depict timelines, drastically reducing planning time.
\end{rSubsubsection}
\vspace{-0.4em}
\end{rSubsection}
\vspace{-0.4em}
\hdashrule[1pt]{19cm}{0.5pt}{1mm 1mm}
\begin{rSubsection}{Relfor Labs Pvt. Ltd.}
{Pune, India}{}{}
\begin{rSubsubsection}{\textbf{Machine Learning Research Scientist (consulting)}}{September 2022 - Present}{}
    \item Set up the ML training pipeline using PyTorch lightning on Nvidia DGX A100, with automatic hyperparameter tuning using Optuna and dynamic architecture updates, expediting experimentation, making it \textbf{$\sim$10\%} faster, which boosted performance metrics.
\end{rSubsubsection}\vspace{-1em}
\begin{rSubsubsection}{\textbf{Machine Learning Engineer}}{August 2021 - November 2021}{}
    \item Designed novel deep Convolutional Neural Network architectures in PyTorch for audio data classification, which beat state-of-the-art models achieving \textbf{>98\%} accuracy and textbf{$\sim$0.98} F1-score.
    \item Implemented various statistical methods using Sci-kit learn to boost model performance by analyzing threshold values and increasing precision to \textbf{$\sim$98\%} while maintaining high accuracy \textbf{>98\%}.
\end{rSubsubsection}
\vspace{-0.4em}
\end{rSubsection}
\vspace{-0.4em}
\hdashrule[1pt]{19cm}{0.5pt}{1mm 1mm}
\begin{rSubsection}{FinIQ Consulting India Pvt. Ltd.}{May 2020 - June 2020}{}{}
\begin{rSubsubsection}{\textbf{Software Development Intern}}{Nashik, India}{}
\item Developed a front-end using AngularJS for forex trading with interactive visualization and chatbot service, providing an appealing platform for forex operations.
\item Created a Python module for stress testing CPU and memory as per user input using variable load calibration.
\item GitHub: \href{https://github.com/obi-wan-shinobi/Stress-test}{CPU and Memory Stressing module} \& \href{https://github.com/obi-wan-shinobi/Forex-Trading}{Forex Trading Platform}.
\end{rSubsubsection}
\end{rSubsection}


\end{rSection}
\smallskip
%	EXAMPLE SECTION
%----------------------------------------------------------------------------------------

%----------------------------------------------------------------------------------------

%----------------------------------------------------------------------------------------
\begin{rSection}{PROJECTS \& RESEARCH}

\begin{rSubsection}{\href{https://github.com/obi-wan-shinobi/Astronomical-Image-Colorization}{Astronomical Image Colorization and Super-resolution using GANs}}{August 2020 - June 2021}{}{}
\vspace{-0.9em}
\item Led a team of four members in a project for automatic colorizing and upscaling low-resolution, grayscale astronomical images.
\item Created a dataset of $\sim 5000$ images by scraping the Hubble archives
\item Developed variations of GAN architectures in Tensorflow, effectively creating a novel training method for colorizing images achieving visually pleasing results
\item Implemented a variation of SRGAN architecture suitable for the data and obtained high-resolution images
\end{rSubsection}

\begin{rSubsection}{\href{https://github.com/obi-wan-shinobi/GalaxyEfficientNets}{The Galaxy Zoo Project}}{August 2019 - September 2020}{}{}
\vspace{-0.9em}
\item A galaxy morphology classification, based on Kaggle Galaxy Zoo 2 competition, implementing the EfficientNet architectures in Tensorflow.
\item Developed a CNN for vote fraction predictions of 37 galaxy features from the Galaxy Zoo decision tree with a RMSE score of \textbf{0.07765}, ranking us in the \textbf{top 3} on the public leaderboard.
\item Developed a CNN for classification of galaxies into 7 classes based on their morphologies with an accuracy of \textbf{93.7\%} and an F1 score of \textbf{0.8857}.
\end{rSubsection}

\begin{rSubsection}{\href{https://github.com/einsteinpy/einsteinpy}{The EinsteinPy Project}}{March 2020 - April 2020}{}{}
\vspace{-0.9em}
\item Contributor to an open source community Python package for general relativity (\textbf{500+ stars} on GitHub).
\item Added Reissner–Nordström metric: a static solution to the Einstein-Maxwell field equations, into the code.
\item Corrections in the Kerr-Newman and Kerr metrics classes.
\item Added calculations of the event horizon and ergosphere for a Kerr-Newman black hole.
\item \href{https://doi.org/10.5281/zenodo.4445219
}{DOI: 10.5281/zenodo.4445219}
\end{rSubsection}

\begin{rSubsection}{Robocon}{August 2018 - May 2020}{}{}
\vspace{-0.9em}
\item Built a quadruped robot with gait similar to a horse, and a wheeled robot with dynamic locomotive abilities for ABU Robocon 2019
\item Mentored junior members of the team for Robocon 2020; planned and assisted in creating two wheeled robots capable of performing intricate tasks of catching and throwing a rugby ball
\end{rSubsection}

\end{rSection}

\begin{rSection}{Scholastic \& Co-curricular Achievements}
\vspace{-1em}
\item Received the Best Outgoing Student Award in 2021 by the Head of Computer Engineering Dept., K. K. Wagh Institute
\item Received the Award of Academic Excellence for the best academic performance across the institution among 1200 students in all Engg. departments
\item Ranked 9 in phase 1 of ABU Robocon 2019 among 200+ teams across the country
\end{rSection}

%----------------------------------------------------------------------------------------
%	TECHNICAL STRENGTHS SECTION
%----------------------------------------------------------------------------------------
\smallskip
\begin{rSection}{Technical Strengths}

\begin{tabular}{ @{} >{\bfseries}l @{\hspace{6ex}} l }
Computer Languages &  C/C++, Python, Ruby, Javascript, Typescript \\
Web Development & AngularJS, VueJS, ElectronJS, Flask, Ruby on Rails, HTML, CSS\\
Deep Learning Frameworks & Keras, TensorFlow, PyTorch \\
Machine Learning Frameworks & Octave, Sci-kit\\
Embedded Systems & Arduino, RaspberryPi, Teensy \\
Version Control & Git, GitHub \\
Tools & Numpy, Pandas, Scipy, \LaTeX

\end{tabular}

\end{rSection}
%----------------------------------------------------

\end{document}    