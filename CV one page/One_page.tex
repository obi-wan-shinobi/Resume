%%%%%%%%%%%%%%%%%
% This is an example CV created using altacv.cls (v1.3, 10 May 2020) written by
% LianTze Lim (liantze@gmail.com), based on the
% Cv created by BusinessInsider at http://www.businessinsider.my/a-sample-resume-for-marissa-mayer-2016-7/?r=US&IR=T
%
%% It may be distributed and/or modified under the
%% conditions of the LaTeX Project Public License, either version 1.3
%% of this license or (at your option) any later version.
%% The latest version of this license is in
%%    http://www.latex-project.org/lppl.txt
%% and version 1.3 or later is part of all distributions of LaTeX
%% version 2003/12/01 or later.
%%%%%%%%%%%%%%%%

%% If you are using \orcid or academicons
%% icons, make sure you have the academicons
%% option here, and compile with XeLaTeX
%% or LuaLaTeX.
% \documentclass[10pt,a4paper,academicons]{altacv}

%% Use the "normalphoto" option if you want a normal photo instead of cropped to a circle
% \documentclass[10pt,a4paper,normalphoto]{altacv}

\documentclass[10pt,a4paper,ragged2e,withhyper]{altacv}

%% AltaCV uses the fontawesome5 and academicon fonts
%% and packages.
%% See http://texdoc.net/pkg/fontawesome5 and http://texdoc.net/pkg/academicons for full list of symbols. You MUST compile with XeLaTeX or LuaLaTeX if you want to use academicons.

% Change the page layout if you need to
\geometry{left=1.25cm,right=1.25cm,top=1.5cm,bottom=1.5cm,columnsep=1.2cm}

% The paracol package lets you typeset columns of text in parallel
\usepackage{paracol}


% Change the font if you want to, depending on whether
% you're using pdflatex or xelatex/lualatex
\ifxetexorluatex
  % If using xelatex or lualatex:
  \setmainfont{Lato}
\else
  % If using pdflatex:
  \usepackage[default]{lato}
\fi

% Change the colours if you want to
\definecolor{arsenic}{rgb}{0.23, 0.27, 0.29}
\definecolor{SlateGrey}{HTML}{2E2E2E}
\definecolor{LightGrey}{HTML}{666666}
\definecolor{Blue}{HTML}{1EA5E8}
\colorlet{heading}{arsenic}
\colorlet{accent}{arsenic}
\colorlet{emphasis}{SlateGrey}
\colorlet{body}{LightGrey}
\colorlet{subheading}{Blue}

% Change some fonts, if necessary
% \renewcommand{\namefont}{\Huge\rmfamily\bfseries}
% \renewcommand{\personalinfofont}{\footnotesize}
% \renewcommand{\cvsectionfont}{\LARGE\rmfamily\bfseries}
% \renewcommand{\cvsubsectionfont}{\large\bfseries}

% Change the bullets for itemize and rating marker
% for \cvskill if you want to
\renewcommand{\itemmarker}{{\small\textbullet}}
\renewcommand{\ratingmarker}{\faCircle}

%% sample.bib contains your publications
\addbibresource{sample.bib}

\begin{document}
\name{Shreyas Kalvankar}
\tagline{}
% Cropped to square from https://en.wikipedia.org/wiki/Marissa_Mayer#/media/File:Marissa_Mayer_May_2014_(cropped).jpg, CC-BY 2.0
%% You can add multiple photos on the left or right
%\photoR{2.5cm}{mmayer-wikipedia-cc-by-2_0}
% \photoL{2cm}{Yacht_High,Suitcase_High}
\personalinfo{%
  % Not all of these are required!
  % You can add your own with \printinfo{symbol}{detail}
  \email{shreyaskalvankar@gmail.com}
  \phone{+919423555723}
  \location{Maharashtra, India}
  \linkedin{linkedin.com/in/shreyas-kalvankar}
  \github{github.com/obi-wan-shinobi}
%   \github{github.com/mmayer} % I'm just making this up though.
%   \orcid{orcid.org/0000-0000-0000-0000} % Obviously making this up too. If you want to use this field (and also other academicons symbols), add "academicons" option to \documentclass{altacv}
  %% You MUST add the academicons option to \documentclass, then compile with LuaLaTeX or XeLaTeX, if you want to use \orcid or other academicons commands.
  % \orcid{0000-0000-0000-0000}
  %% You can add your own arbtrary detail with
  %% \printinfo{symbol}{detail}[optional hyperlink prefix]
  % \printinfo{\faPaw}{Hey ho!}
  %% Or you can declare your own field with
  %% \NewInfoFiled{fieldname}{symbol}[optional hyperlink prefix] and use it:
  % \NewInfoField{gitlab}{\faGitlab}[https://gitlab.com/]
  % \gitlab{your_id}
}

\makecvheader

%% Depending on your tastes, you may want to make fonts of itemize environments slightly smaller
\AtBeginEnvironment{itemize}{\small}

%% Set the left/right column width ratio to 6:4.
\columnratio{0.6}

% Start a 2-column paracol. Both the left and right columns will automatically
% break across pages if things get too long.
\begin{paracol}{2}

\cvsection{Experience}

\cvevent{Summer Intern}{FinIQ Consulting India Pvt. Ltd.}{May 2020 - June 2020}{Nashik, India}
\begin{itemize}
\item Set up an online platform for Forex trading and essential services such as market news, chatbot, etc using AngularJS
\item Created a python module for stress testing CPU and memory with variable load
\end{itemize}

\cvsection{Technical Skills}
\cvskill{C/C++, Python, Java}{4}
\cvskill{Deep Learning}{4}
\cvskill{Computer Vision}{4}
\cvskill{Machine Learning}{3}
\cvskill{Databases}{3}
\cvskill{Robotics}{3}
\cvskill{Web Development}{2}

\cvsection{Software Skills}
\begin{itemize}
    \item \textbf{Python Libraries} : Tensorflow, keras, pandas, numpy, matplotlib
    \item \textbf{C++} : Generic programming, Standard Template Libraries
    \item \textbf{Deep Learning} : Image recognition and classification, time series analysis, Natural Language Processing
    \item \textbf{Tools} : Git, Octave
    \item \textbf{Embedded Software Programming} : Arduino, Raspberry Pi, Teensy
\end{itemize}

\cvsection{Positions of Responsibility}
\cvevent{Software Developer}{Team Vector, ABU Robocon 2019}{August 2018 - April 2019}{}
\begin{itemize}
    \item Helped build and develop a code for an autonomous quadruped robot
\end{itemize}

\divider

\cvevent{Mentor}{Team Vector, ABU Robocon 2020}{August 2019 - April 2020}{}
\begin{itemize}
    \item Helped and guided junior members of the team in building a omni-wheeled robot
\end{itemize}

\cvsection{Strengths}
\cvtag{Curious}
\cvtag{Hardworking}
\cvtag{Adaptable}
\cvtag{Communication skills}


%% Switch to the right column. This will now automatically move to the second
%% page if the content is too long.
\switchcolumn

\cvsection{Education}
\cvevent{B.E (Computer Engineering)}{K.K. Wagh Institute of Engineering Education and Research}{2017-2021}{Nashik}
\begin{itemize}
    \item CGPA: 9.54/10
\end{itemize}

\cvevent{Higher Secondary Certificate}{H.P.T Arts and R.Y.K Science College}{2017}{Nashik}
\begin{itemize}
    \item 87.07\%
\end{itemize}

\cvevent{Secondary School Certificate}{Boys' Town Public School}{2015}{Nashik}
\begin{itemize}
    \item 94.4\%
\end{itemize}

\cvsection{Projects \& Research}
\cvsubsection{THE GALAXY ZOO PROJECT}
\begin{itemize}
    \item Studied galaxy morphology classification and CNN architectures
    \item Developed a CNN architecture for the vote fraction predictions of 37 features in the galaxy zoo decision tree
    \item Developed a network for classification of galaxies into seven morphologies
\end{itemize}
\cvsubsection{THE EINSTEINPY PROJECT}
\begin{itemize}
    \item An open source community python package for general relativity
    \item \textbf{Contributions}:
    \begin{itemize}
        \item Addition of  Reissner–Nordström metric: a static solution to the Einstein-Maxwell field equations (\textit{PR: \#462 Issue: \#309})
        \item Correction in the Kerr-Newman and Kerr metrics
        \item Added calculations of event horizon and ergosphere for a Kerr-Newman blackhole (\textit{PR: \#472 Issue: \#109}) 
    \end{itemize}
\end{itemize}
\cvsection{Publications}
\nocite{*}
\printbibliography[heading=pubtype,title={\printinfo{\faFile*[regular]}{Journal Articles}}, type=misc]
\end{paracol}

\end{document}
