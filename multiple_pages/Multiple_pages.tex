%%%%%%%%%%%%%%%%%%%%%%%%%%%%%%%%%%%%%%%%%
% Medium Length Professional CV
% LaTeX Template
% Version 2.0 (8/5/13)
%
% This template has been downloaded from:
% http://www.LaTeXTemplates.com
%
% Original author:
% Trey Hunner (http://www.treyhunner.com/)
%
% Important note:
% This template requires the resume.cls file to be in the same directory as the
% .tex file. The resume.cls file provides the resume style used for structuring the
% document.
%
%%%%%%%%%%%%%%%%%%%%%%%%%%%%%%%%%%%%%%%%%

%----------------------------------------------------------------------------------------
%	PACKAGES AND OTHER DOCUMENT CONFIGURATIONS
%----------------------------------------------------------------------------------------

\documentclass{resume} % Use the custom resume.cls style

\usepackage{hyperref}
\usepackage[left=0.75in,top=0.6in,right=0.75in,bottom=0.6in]{geometry} % Document margins

  % If using pdflatex:
\usepackage[utf8]{inputenc}
\usepackage[T1]{fontenc}
\usepackage[default]{lato}
\usepackage{fontawesome}
\usepackage{setspace}
\usepackage{biblatex}

\usepackage{hyperref}
\hypersetup{
    colorlinks=true,
    linkcolor=blue,
    filecolor=magenta,
    urlcolor=cyan,
}


\newcommand{\tab}[1]{\hspace{.2667\textwidth}\rlap{#1}}
\newcommand{\itab}[1]{\hspace{0em}\rlap{#1}}
\name{Shreyas Kalvankar} % Your name
\address{\faMapMarker ~ Nashik, Maharashta, India} % Your address
%\address{123 Pleasant Lane \\ City, State 12345} % Your secondary addess (optional)
\address{\footnotesize \faPhone ~ (+91)~9423555723 ~ \faEnvelope ~ shreyaskalvankar@gmail.com ~ \faLinkedin ~ linkedin.com/in/shreyas-kalvankar ~ \faGithub ~ github.com/obi-wan-shinobi} % Your phone number and email

\addbibresource{references.bib}

\begin{document}

%----------------------------------------------------------------------------------------
%	EDUCATION SECTION
%----------------------------------------------------------------------------------------

\begin{rSection}{Education}

{\bf Bachelor of Engineering(Computer Engineering)} \hfill {\em 2017 - 2021}
\\ K.K. Wagh Institute of Engineering  \hfill { Overall GPA: 7.6/10}
\\ Education \& Research, Nashik

\smallskip

{\bf Higher Secondary Certificate} \hfill {\em 2017}
\\ HPT Arts \& RYK Science College, Nashik \hfill { Percentage: 87.07\%}

\smallskip

{\bf Secondary School Certificate} \hfill {\em 2015}
\\ Boys' Town Public School, Nashik \hfill { Percentage: 94.4\% }

\end{rSection}
%----------------------------------------------------------------------------------------
%	TECHNICAL STRENGTHS SECTION
%----------------------------------------------------------------------------------------

\begin{rSection}{Technical Strengths}

\begin{tabular}{ @{} >{\bfseries}l @{\hspace{6ex}} l }
Computer Languages &  C/C++, Python, Java \\
Web Development & AngularJS, Typescript\\
Deep Learning Frameworks & Keras, TensorFlow \\
Machine Learning Frameworks & Octave, Sci-kit\\
Embedded Systems & Arduino, RaspberryPi, Teensy \\
Version Control & Git, GitHub \\

\end{tabular}

\end{rSection}

%----------------------------------------------------------------------------------------
%	WORK EXPERIENCE SECTION
%----------------------------------------------------------------------------------------

\begin{rSection}{Experience}

\begin{rSubsection}{FinIQ Consulting India Pvt. Ltd.}{May 2020 - June 2020}{Summer Intern}{}
\item Studied technical analysis of option chain, equity derivatives
\item Developed a front-end platform using AngularJS for forex trading with history charts, exhange rates and along with a news portal and chatbot service
\item Studied and analysed data cubes and OLAP for business intelligence using company platforms
\item Created a python module for stress testing CPU and memory as per user input using variable load calibration
\item Documented relevant codes and procedure
\item GitHub: \href{https://github.com/obi-wan-shinobi/Stress-test}{CPU and Memory Stressing module} \& \href{https://github.com/obi-wan-shinobi/Forex-Trading}{Forex Trading Platform}
\end{rSubsection}


\end{rSection}


%	EXAMPLE SECTION
%----------------------------------------------------------------------------------------

%----------------------------------------------------------------------------------------
\begin{rSection}{Relevant Courses}
\itab{\textbf{Core Courses}} \tab{}  \tab{\textbf{MOOC}}
\\ \itab{Data Structures and Algorithms} \tab{}  \tab{Deep Learning}
\\ \itab{Computer Organization} \tab{}  \tab{Machine Learning}
\\ \itab{Operating Systems} \tab{}  \tab{Computer Vision}
\\ \itab{Theory of Computation} \tab{} \tab{Tensorflow and Keras}
\\ \itab{Database Management Systems} \tab{}
%\\ \itab{Data Analysis (Ongoing)} \tab{} \tab{Electrodynamics}
%\\ \itab{Data Mining and Warehousing (Ongoing)}
\newline
\newline
\itab{\textbf{Other Relevant Courses}}
\\ \itab{Introduction to General Theory of Relativity}
\\ \itab{Linear Algebra}
\\ \itab{Mathematics for Machine Learning}


\end{rSection}

\pagebreak

\begin{rSection}{POSITIONS OF RESPONSIBILITY}

\begin{rSubsection}{Team Vector}{August 2018 - April 2019}{Developer}{ABU Robocon 2019}
\item Assigned to build and code a quadruped robot and a wheeled robot with dynamic locomotive abilities
\item The project was about an annual competition conducted by Asia Broadcast Union and consisted of a series of tasks that were supposed to be performed abiding the rules of the competition
\item Two robots were created, one being an autonomous quadruped and the other a wheeled robot which had dynamic locomotive abilities
\end{rSubsection}

%------------------------------------------------

\begin{rSubsection}{Team Vector}{August 2019 - April 2020}{Mentor}{ABU Robocon 2020}
\item Mentored junior members of the team for designing two robots with holonomic drives
\item The project was about a competition which would have the robots play rugby with 5 obstacles in the way
\item Two robots were created out of which one was supposed to have throwing and kicking capabilities and the other was supposed to have catching and placing capabilities. Both robots had dynamic locomotive abilities owing to the holonomic drive design
\end{rSubsection}

%------------------------------------------------

\end{rSection}



%----------------------------------------------------------------------------------------
\begin{rSection}{PROJECTS \& RESEARCH}

\begin{rSubsection}{\href{https://github.com/obi-wan-shinobi/Galaxy_Zoo}{The Galaxy Zoo Project}}{August 2019 - Present}{}{}
\item A galaxy morphology classification project using deep learning
\item Studied different convolutional neural networks and their architectures
\item Studied different architectural blocks to enhance performance
\item Developed a network for vote fraction predictions of 37 galaxy features from the Galaxy Zoo decision tree
\item Developed a network for classification of galaxies into 7 classes based on their morphologies
\end{rSubsection}

\begin{rSubsection}{\href{https://github.com/obi-wan-shinobi/einsteinpy}{The EinsteinPy Project}}{March 2020 - April 2020}{}{}
\item Contributer to an open source community python package for general relativity
\item \textbf{Contributions}:
\item Addition of  Reissner–Nordström metric: a static solution to the Einstein-Maxwell field equations, into the code (\textit{PR: \#462 Issue: \#309})
\item Corrections in the Kerr-Newman and Kerr metrics classes
\item Added calculations of event horizon and ergosphere for a Kerr-Newman blackhole (\textit{PR: \#472 Issue: \#109})
\end{rSubsection}

\begin{rSubsection}{\href{https://github.com/obi-wan-shinobi/BTC_predictor}{Time series analysis and prediction}}{March 2019}{}{}
\item Developed a recurrent neural network that analyses time series and predicts future time frame
\item The project took into account a stock price, bitcoin exchange and other time series and could predict almost accurately the trend in prices
\item Another project consisted of using pandemic data of active cases and visualising them as a time series and predicting the epi-curve for COVID-19
\item An introductory project for LSTM networks which are extensively used in audio and sound analysis
\end{rSubsection}

\pagebreak

\begin{rSubsection}{\href{https://github.com/obi-wan-shinobi/kuzushiji}{Kuzushiji Recognition}}{September 2019}{}{}
\item A Kaggle competition to transcribe ancient Kuzushiji into contemporary Japanese characters
\item Created a code to visualize the data and performed statistical analysis
\item Built a model to recognize the handwritten text
\end{rSubsection}

\begin{rSubsection}{Deep Writing}{December 2019}{}{}
\item Created a RNN model with LSTM blocks to analyse books
\item Trained the network over books from the same author and generated portions of new text
\end{rSubsection}

\begin{rSubsection}{Natural Language Processing}{December 2019 - May 2020}{}{}
\item Made various short projects relating to Natural Language Processing
\item Created a RNN model and trained it over jokes dataset to generate jokes
\item Created a RNN \& LSTM network model and trained it over a poem dataset to generate poems
\item Created and trained an ngram model and trained it over twitter data to generate tweets
\end{rSubsection}


\end{rSection}



%--------------------------------------------------
\begin{rSection}{PUBLICATIONS}
\item Shreyas Bapat et al. \textit{EinsteinPy: A Community Python Package for General Relativity.} 2020.\\ arXiv: \href{https://arxiv.org/abs/2005.11288}{2005.11288 [gr-qc]}.

\end{rSection}



%----------------------------------------------------
\begin{rSection}{LANGUAGES}
\item English \textit{\footnotesize{(Native or bilingual proficiency)}}
\item Hindi \textit{\footnotesize{(Native or bilingual proficiency)}}
\item Marathi \textit{\footnotesize{(Native or bilingual proficiency)}}
\item Sanskrit \textit{\footnotesize{(Limited working proficiency)}}
\item Japanese \textit{\footnotesize{(Elementary proficiency)}}
\end{rSection}

%---------------------------------------------------
\begin{rSection}{INTERESTS}
\item Deep Learning
\item Computer Vision
\item Linear Algebra
\item Machine Learning
\item Data Structures \& Algorithms
\item Data Science
\item Differential Geometry \& General Relativity
\end{rSection}


\end{document}
