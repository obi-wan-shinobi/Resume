%%%%%%%%%%%%%%%%%
% This is an example CV created using altacv.cls (v1.3, 10 May 2020) written by
% LianTze Lim (liantze@gmail.com), based on the
% Cv created by BusinessInsider at http://www.businessinsider.my/a-sample-resume-for-marissa-mayer-2016-7/?r=US&IR=T
%
%% It may be distributed and/or modified under the
%% conditions of the LaTeX Project Public License, either version 1.3
%% of this license or (at your option) any later version.
%% The latest version of this license is in
%%    http://www.latex-project.org/lppl.txt
%% and version 1.3 or later is part of all distributions of LaTeX
%% version 2003/12/01 or later.
%%%%%%%%%%%%%%%%

%% If you are using \orcid or academicons
%% icons, make sure you have the academicons
%% option here, and compile with XeLaTeX
%% or LuaLaTeX.
% \documentclass[10pt,a4paper,academicons]{altacv}

%% Use the "normalphoto" option if you want a normal photo instead of cropped to a circle
% \documentclass[10pt,a4paper,normalphoto]{altacv}

\documentclass[10pt,a4paper,ragged2e,withhyper]{altacv}

%% AltaCV uses the fontawesome5 and academicon fonts
%% and packages.
%% See http://texdoc.net/pkg/fontawesome5 and http://texdoc.net/pkg/academicons for full list of symbols. You MUST compile with XeLaTeX or LuaLaTeX if you want to use academicons.

% Change the page layout if you need to
\geometry{left=1.1cm,right=1.1cm,top=1.2cm,bottom=1.2cm,columnsep=0.8cm}

% The paracol package lets you typeset columns of text in parallel
\usepackage{paracol}


% Change the font if you want to, depending on whether
% you're using pdflatex or xelatex/lualatex
\ifxetexorluatex
  % If using xelatex or lualatex:
  \setmainfont{Lato}
\else
  % If using pdflatex:
  \usepackage[default]{lato}
\fi

% Change the colours if you want to
\definecolor{arsenic}{rgb}{0.23, 0.27, 0.29}
\definecolor{SlateGrey}{HTML}{2E2E2E}
\definecolor{LightGrey}{HTML}{666666}
\definecolor{Blue}{HTML}{1EA5E8}
\colorlet{heading}{arsenic}
\colorlet{accent}{arsenic}
\colorlet{emphasis}{SlateGrey}
\colorlet{body}{LightGrey}
\colorlet{subheading}{Blue}

% Change some fonts, if necessary
% \renewcommand{\namefont}{\Huge\rmfamily\bfseries}
% \renewcommand{\personalinfofont}{\footnotesize}
% \renewcommand{\cvsectionfont}{\LARGE\rmfamily\bfseries}
% \renewcommand{\cvsubsectionfont}{\large\bfseries}

% Change the bullets for itemize and rating marker
% for \cvskill if you want to
\renewcommand{\itemmarker}{{\small\textbullet}}
\renewcommand{\ratingmarker}{\faCircle}
\newcommand*\googlescholaricon{\raisebox{-0.3em}{\includegraphics[width=1em]{google-scholar.pdf}}}

%% sample.bib contains your publications
\addbibresource{sample.bib}

\begin{document}
\name{Shreyas Kalvankar}
\tagline{}
%% You can add multiple photos on the left or right
%\photoR{2.5cm}{mmayer-wikipedia-cc-by-2_0}
% \photoL{2cm}{Yacht_High,Suitcase_High}
\personalinfo{%
  % Not all of these are required!
  % You can add your own with \printinfo{symbol}{detail}
  \printinfo{\faGlobe}{obi-wan-shinobi.github.io}
  \email{shreyaskalvankar@gmail.com}
  \phone{+919423555723}
  % \location{Maharashtra, India}
  \linkedin{linkedin.com/in/shreyas-kalvankar}
  \github{github.com/obi-wan-shinobi}
  \printinfo{\googlescholaricon}{Scholar Profile \href{https://scholar.google.com/citations?user=W8-T__UAAAAJ&hl=en}{[link]}}
%   \github{github.com/mmayer} % I'm just making this up though.
%   \orcid{orcid.org/0000-0000-0000-0000} % Obviously making this up too. If you want to use this field (and also other academicons symbols), add "academicons" option to \documentclass{altacv}
  %% You MUST add the academicons option to \documentclass, then compile with LuaLaTeX or XeLaTeX, if you want to use \orcid or other academicons commands.
  % \orcid{0000-0000-0000-0000}
  %% You can add your own arbtrary detail with
  %% \printinfo{symbol}{detail}[optional hyperlink prefix]
  % \printinfo{\faPaw}{Hey ho!}
  %% Or you can declare your own field with
  %% \NewInfoFiled{fieldname}{symbol}[optional hyperlink prefix] and use it:
  % \NewInfoField{gitlab}{\faGitlab}[https://gitlab.com/]
  % \gitlab{your_id}
}

\makecvheader

%% Depending on your tastes, you may want to make fonts of itemize environments slightly smaller
\AtBeginEnvironment{itemize}{\small}

%% Set the left/right column width ratio to 6:4.
\columnratio{0.6}

% Start a 2-column paracol. Both the left and right columns will automatically
% break across pages if things get too long.
\begin{paracol}{2}

\cvsection{Experience}
\cvevent{Software Developer}{Dalton Maag}{November 2021 - Present}{London, United Kingdom}
\begin{itemize}
    \item Assisted in developing a Python system that used Genetic Algorithms for automatically generating thousands of CJK font glyphs, drastically cutting down production time.
    \item \textbf{Pricebot}: Pricebot simplified typeface pricing by considering factors like the number of weights, axes, and scripts. It aimed to ensure accurate and consistent pricing while relieving designers of the time-consuming and error-prone task of manual quoting, thus preventing project overruns and unexpected costs.
    \item It's a web app with a Ruby on Rails back-end and a VueJS \& Typescript front-end. I developed and fine-tuned the pricing models in Typescript based on expected outputs.
    \item My colleague and I enhanced glyph data models for Arabic, Greek, Cyrillic scripts, accurately representing their letters. I also created a new model from scratch for Devanagari, ensuring precise pricing for non-Latin projects.
    \item Using graph theory, I devised a process to efficiently create project plans that accurately depict timelines, drastically reducing planning time.
\end{itemize}
\divider
\cvevent{Machine Learning Research Scientist (Consulting)}{Relfor Labs Pvt. Ltd.}{September 2022 - Present}{Pune, India}
\begin{itemize}
    \item Set up the ML training pipeline using PyTorch lightning on Nvidia DGX A100, with automatic hyperparameter tuning using Optuna and dynamic architecture updates, expediting experimentation, making it \textbf{$\sim$10\%} faster, which boosted performance metrics.
\end{itemize}
\cvevent{Machine Learning Engineer}{Relfor Labs Pvt. Ltd.}{August 2021 - November 2021}{Pune, India}
\begin{itemize}
    \item Worked on audio data classification and designed multiple novel deep convolutional neural network architectures in PyTorch, which beat state-of-the-art models with \textbf{~98.6\%} accuracy and \textbf{$\sim$0.98} F1-score
   \item Leveraged metric analysis techniques in SKLearn and PyTorch to determine optimum threshold values, achieving a precision of \textbf{$\sim$98\%} while maintaining high accuracy \textbf{>98\%}.
\end{itemize}
\divider
\cvevent{Software Development Intern}{FinIQ Consulting India Pvt. Ltd.}{May 2020 - June 2020}{Nashik, India}
\begin{itemize}
\item Set up an online platform for Forex trading using AngularJS as a new feature for the customers
\item Created a python module for stress testing CPU and memory with variable load for integration in the company cloud platforms' testing pipeline
\end{itemize}

\begin{comment}
\cvsection{Technical Skills}
\cvskill{C/C++, Python, Java}{4}
\cvskill{Deep Learning}{4}
\cvskill{Computer Vision}{4}
\cvskill{Machine Learning}{3}
\cvskill{Databases}{3}
\cvskill{Robotics}{3}
\cvskill{Web Development}{2}
\end{comment}


\cvsection{Technical Skills}
\begin{itemize}
    \item \textbf{Computer Languages} : C, C++, Python, 
    \item \textbf{Web Development} : VueJS, Javascript, Typescript, HTML, CSS, Ruby on Rails
    \item \textbf{ML Frameworks} : Keras, Tensorflow, PyTorch, Sci-Kit Learn
    % \item \textbf{Machine Learning Frameworks} : Octave, Sci-kit Learn
    % \item \textbf{Embedded Software Programming} : Arduino, Raspberry Pi, Teensy
    % \item \textbf{Version Control} : Git, GitHub
\end{itemize}

% \cvsection{Positions of Responsibility}
% \cvevent{Software Developer}{Team Vector, ABU Robocon 2019}{August 2018 - April 2019}{}
% \begin{itemize}
%     \item Assigned to build and code a quadruped robot and a wheeled robot with dynamic locomotive abilities for ABU Robocon 2019
% \end{itemize}

% \divider

% \cvevent{Mentor}{Team Vector, ABU Robocon 2020}{August 2019 - April 2020}{}
% \begin{itemize}
%     \item Helped and guided junior members of the team in building robots that could efficiently handle locomotion and throwing, catching and kicking a football
% \end{itemize}

\begin{comment}
\cvsection{Strengths}
\cvtag{Deep Learning}
\cvtag{Machine Learning \& Data Science}\\
\cvtag{Software Development}
\cvtag{Researcher}
\cvtag{Hardworking}\\
\cvtag{Adaptable}
\cvtag{Communication skills}
\end{comment}
%% Switch to the right column. This will now automatically move to the second
%% page if the content is too long.
\switchcolumn

\cvsection{Education}
\cvevent{B.E (Computer Engineering)}{K.K. Wagh Institute of Engineering Education and Research}{2017-2021}{Nashik}
\begin{itemize}
    \item CGPA: 9.7/10 (Rank 1)
\end{itemize}

\cvsection{Personal Projects \& Research}

    \cvsubsection{THE GALAXY ZOO PROJECT}
    \begin{itemize}
        \item Developed a CNN in Tensorflow for vote fraction predictions of 37 galaxy features from the Galaxy Zoo decision tree with an rmse score of \textbf{0.07765}, ranking us in the \textbf{top 3} on the public leaderboard
        \item Also developed a CNN for classification of galaxies into 7 classes based on their morphologies with an accuracy of \textbf{93.7\%} and an F1 score of \textbf{0.8857}
    \end{itemize}
\begin{comment}
\cvsubsection{THE SCHRODPY PROJECT}
    \begin{itemize}
        \item The project is in early phases, which plans at providing an efficient simulation of quantum systems
        \item Applied various methods to numerically solve Schrödinger's Time Independent wave equation for finding stationary states of the particle in an Infinite Potential Well
        \item Prospects: An efficient and community python package for simulating quantum many-body systems using tensor networks
    \end{itemize}
\end{comment}
\cvsubsection{THE EINSTEINPY PROJECT}
\begin{itemize}
    \item An open source community python package for general relativity
    \item \textbf{Contributions}:
    \begin{itemize}
        \item Added Reissner–Nordström metric: a static solution to the Einstein-Maxwell field equations, into the code %(\textit{PR: \#462 Issue: \#309})
        \item Corrections in the Kerr-Newman and Kerr metrics classes
        \item Added calculations of event horizon and ergosphere for a Kerr-Newman blackhole %(\textit{PR: \#472 Issue: \#109}) 
        \item \href{https://doi.org/10.5281/zenodo.4445219}{DOI: 10.5281/zenodo.4445219}
    \end{itemize}
\end{itemize}
\cvsection{Publications}
\nocite{*}
\printbibliography[heading=pubtype,title={\printinfo{\faFile*[regular]}{Journal Articles}}, type=misc]


\end{paracol}

\end{document}
